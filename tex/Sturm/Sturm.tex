% Sturm.tex
% Copyright (c) Sergey Strukov. All rights reserved. This is a public document. You can freely distribute and use it, providing the authorship and the copyright note is unchanged.

\input ../SSCommon.tex

\begin{document}
\selectlanguage{russian}
    
\SScover
    
\SStitle{Теорема Штурма}

Теорема Штурма --- красивая элементарная теорема школьного уровня. Она позволяет найти число корней данного полинома с вещественными коэффициентами на заданном интервале.
С её помощью можно локализовать корни вещественных полиномов и находить хорошие приближения к ним.

\vspace

\SSbullet 

\begin{tikzpicture}
    \draw[thick,->] (0,-2) -- (0,2) node[anchor=west] {\(y\)};
    \draw[thick,->] (-2,0) -- (2,0) node[anchor=north] {\(x\)};
    \draw (1,1) node[anchor=south west] {\(0\)};
    \draw (1,-1) node[anchor=north west] {\(1\)};
    \draw (-1,1) node[anchor=south east] {\(1\)};
    \draw (-1,-1) node[anchor=north east] {\(0\)};
\end{tikzpicture}

\SSsect[def] Функция \( \sigma: \xR^* \times \xR^* \rightarrow \{0,1\} \) определена как
\[ \sigma(x,y) = 
   \begin{cases} 
       1, & sign(x) = sign(y) \\ 
       0, & sign(x) \neq sign(y)
   \end{cases} 
\]

\SSsect \( \sigma(x,y) = \sigma(y,x) \)

\SSsect \( \sigma(-x,y) = 1 - \sigma(x,y) \)

\SSsect[def] \( \sigma(x_1,\dots,x_n) \) , \( x_1,\dots,x_n \in \xR^* \) , \( n \geqslant 1 \)
\[ \sigma(x_1,\dots,x_n) = \sum_{k=1}^{n-1} \sigma(x_k,x_{k+1})
\]
--- число перемен знака (Ч.П.З.)

\SSsect Индуктивно,
\begin{itemize}[label=]
\item \( \sigma(x_1) = 0 \) ,
\item \( \sigma(x_1,\dots,x_n) = \sigma(x_1,x_2) + \sigma(x_2,\dots,x_n) \) , \( n \geqslant 2 \) .
\end{itemize}
        

\end{document}

