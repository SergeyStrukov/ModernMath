% Limits.tex
% Copyright (c) Sergey Strukov. All rights reserved. This is a public document. You can freely distribute and use it, providing the authorship and the copyright note is unchanged.

\input ../SSCommon.tex

\begin{document}
\selectlanguage{russian}
    
\SScover
    
\SStitle{Пределы}

Исторически, теория пределов возникла как часть анализа. 
Однако, после появления топологии она в обобщённой форме стала естественной частью топологии. 
В этой статье теория пределов излагается в законченной геометрической форме. 
Кратко говоря, предел --- это продолжение функции по непрерывности на специальных топологических пространствах --- \mbox{\textbf{фильтрах}}. 
Подобная конструкция делает большинство свойств пределов \textit{наглядно очевидными}.

\vspace
    
\SSbullet 

\SSsect[def] Пусть \( F \) --- топологическое пространство. 
\( F \) называется \underline{фильтром}, если все точки \( F \), кроме одной, открыты. 
Такая точка, тавтологически, определена однозначно. 
Будем обозначать её \( \infty_F \). 
Следуя общему правилу, положим \( F^\circ := F\backslash\{\infty_F\} \).

\SSsect Пусть \( F \) --- \underline{фильтр}. 

Тогда справедливы следующие утверждения:
\begin{itemize}[label=]
\item \( F^\circ \) открыто, но не замкнуто,
\item \( F^\circ \) дискретно,
\item \( \overline{F^\circ} = F ~,~ F^\circ\neq\varnothing \) ,
\item \( \infty_F \) замкнута,
\item \( F \) недискретно.
\end{itemize}

\SSsect[def] Пусть \( F \) --- фильтр, \( X \) --- топологическое пространство, 
\mbox{\( f: F^\circ \rightarrow X \) --- отображение}. 

Тогда \( f \) непрерывно.

Пусть \( x \in X \), \( g \) --- продолжение \( f \) на \( F \), такое, что \( g(\infty_F) = x \) .

Если \( g \) непрерывно, то будем говорить, что \( \displaystyle x = \lim_{F} f\) .

\begin{tikzpicture}
\SSDiag[M]
{ 
    F^\circ & F & \infty_F & g(\infty_F) = x \\
    & X \\
};
\path
(M-1-1) edge [->] node [above] {\(\subset\)} (M-1-2)
(M-1-1) edge [->] node [left] {\(f\)} (M-2-2)
(M-1-2) edge [->,dashed] node [right] {\(g\)} (M-2-2)
(M-1-3) edge [->] (M-1-2) 
(M-1-3) edge [->] node [right] {\(x\)} (M-2-2) ;
\end{tikzpicture}

\SSsect Пусть \( F \) --- фильтр, \( p: X \rightarrow Y \) --- непрерывное отображение топологических пространств, \( f: F^\circ \rightarrow X \) --- отображение.

Тогда \( \displaystyle x = \lim_{F} f ~\Rightarrow~ \displaystyle p(x) = \lim_{F} p\circ f \) .

\SSsect Пусть \( F \) --- фильтр, \( X \) --- топологическое пространство, \( \left\lbrace Y_i \right\rbrace_{i \in I} \) --- семейство топологических пространств, \( \left\lbrace p_i : X \rightarrow Y_i \right\rbrace_{i \in I} \) --- семейство непрерывных отображений. 
Пусть топология \( X \) порождена семейством \( \left\lbrace p_i \right\rbrace_{i \in I} \) . 
Пусть \(  f: F^\circ \rightarrow X ~,~ x \in X \) .

Тогда, \( \displaystyle x = \lim_{F} f ~\Leftrightarrow~ \forall~i \in I~ p_i(x) = \lim_{F} p_i \circ f \) . 

\SSsect Пусть \( F \) --- фильтр, \( X \) --- топологическое пространство, 
\( f: F^\circ \rightarrow X \) --- отображение.

Пусть \( A \subset X \) замкнуто.

Если \( f\left( F^\circ \right) \subset A ~,~ \displaystyle x = \lim_{F} f \) , то \( x \in A \) .

\SSproof

Пусть \( g \) --- непрерывное продолжение \( f \), такое, что \( g(\infty_F) = x \) .

Тогда \( x \in g(F) = g(\overline{F^\circ}) \subset \overline{g(F^\circ)} = \overline{f(F^\circ)} \subset \overline{A} = A \) .

Тоже самое по-другому: \( F^\circ \subset g^{-1}(A) ~\Rightarrow~ \overline{F^\circ} \subset g^{-1}(A) ~\Rightarrow~ \infty_F \in g^{-1}(A) ~\Rightarrow~ x \in A \) . 

\SSendp

\SSsect[!!!] Пусть \( F \) --- фильтр, \( X \) --- \textbf{хаусдорфово} топологическое пространство, \( f: F^\circ \rightarrow X \) .

Тогда \( \exists \) не более одного предела \( \displaystyle x = \lim_{F} f \) .

Почему у этого пункта аж три восклицательных знака? 
Причина в том, что в математике очень часто пределы в хаусдорфовы пространства используются для \textbf{построения} объектов. 
Например: производная, интеграл, сумма ряда и.т.п. 
Для этого нужна единственность, обеспеченная этим свойством.

\SSproof

Пусть \( \displaystyle x,y = \lim_{F} f \) . 
Тогда \( \displaystyle (x,y) = \lim_{F}~(f,f) \) , но \( (f,f) : F^\circ \rightarrow X \times X \) отображает \( F^\circ \) в диагональ \( \Delta_X \subset X \times X \), которая, в силу хаусдорфовости, замкнута в \( X \times X \). 
Значит, \( (x,y) \in \Delta_X \) , т.е. \( x=y \) .

\SSendp

\pagebreak

\SSbullet 

\SSsect[def] Пусть \( F \) и \( G \) --- два фильтра. 
\( \tau : F \rightarrow G \) --- \underline{морфизм фильтров}, если \( \tau \) --- непрерывное отображение и \( \tau^{-1}(\infty_G)=\{\infty_F\} \).

Фильтры и их морфизмы образуют категорию.

\SSsect Пусть \( F \) и \( G \) --- два фильтра. Если \( \tau : F \rightarrow G \) (морфизм фильтров), то:
\begin{itemize}[label=]
\item \( \tau(F^\circ) \subset G^\circ \) ,
\item \( \tau(\infty_F)=\infty_G \) , 
\item \( \tau^\circ: F^\circ \rightarrow G^\circ \) --- отображение, индуцированное \( \tau \) .
\end{itemize}

\SSsect Пусть \( F \) и \( G \) --- два фильтра.
Пусть \( \tau^\circ: F^\circ \rightarrow G^\circ \) --- отображение. 
Тогда \(  \tau^\circ \) может быть продолжено до морфизма фильтров \( \tau : F \rightarrow G \) (очевидно, единственным образом) \( \Leftrightarrow~ \displaystyle \infty_G = \lim_{F} \tau^\circ \) .

\SSsect Пусть \( F \) и \( G \) --- два фильтра, \( X \) --- топологическое пространство, \( \tau : F \rightarrow G \) , \( f:G^\circ \rightarrow X \) .

Тогда если \( \displaystyle x = \lim_{G} f \) , то \( \displaystyle x = \lim_{F} f\circ \tau^\circ \) .

\pagebreak

\SSbullet 

\SSsect Пусть \( G \) --- фильтр, \( F \) --- топологическое пространство, \( \tau : F \rightarrow G \) --- непрерывное инъективное отображение.

Тогда \( F \) дискретно \( \Leftrightarrow~ \tau^{-1}(\infty_G) \) открыто.

Если \( F \) недискретно, то \( F \) --- фильтр и \( \tau \) --- морфизм фильтров. 

\SSproof

\SSendp

\SSsect
\SSsect
\SSsect
\SSsect

\end{document}
