 % Convention.tex
% Copyright (c) Sergey Strukov. All rights reserved. This is a public document. You can freely distribute and use it, providing the authorship and the copyright note is unchanged.

\input ../SSCommon.tex

\begin{document}
\selectlanguage{russian}

\SScover

\SStitle{Удобства}

\SSbullet

\SSsect Многие математические объекты есть множества с дополнительными структурами.
Довольно часто в этих структурах есть один выделенный элемент базового множества. 
Например, если \( G \) --- группа, то этот выделенный элемент --- единица группы. 
В таком случае удобно символом \( G^\circ \) обозначать базовое множество с выколотым выделенным элементом.
Например, \( \xZ^\circ \) --- множество ненулевых целых чисел.

\SSsect Пусть дано топологическое пространство \( X \) и его точка \( x \). Говорят, что эта точка замкнута (открыта, и.т.п.), если одноточечное множество \( \{ x \} \) замкнуто (соответственно открыто, и.т.п.) в \( X \).

\SSsect Несколько полезных определений:
\[ \xR_+ := \SSet{ t \in \xR }{ t>0 } \]
\[ \xR_- := \SSet{ t \in \xR }{ t<0 } \]
\[ \xZ_+ := \SSet{ t \in \xZ }{ t>0 } \]
\[ \xZ_- := \SSet{ t \in \xZ }{ t<0 } \]

\SSsect Если \( p \) --- простое число, то \( \xF \) --- простое (конечное) поле из \( p \) элементов.

\SSsect \( \xSign \) --- группа знаков, стандартная группа второго порядка.

\SSsect Одномерный тор: 
\[ \xT := \SSet{ z \in \xC }{ |z|=1 } \]

Одна из важнейших топологических групп во всей математике.

\SSsect Стандартная комплексная синусоида:
\[ \xe(t) = e^{2 \pi i t} ~,~ t \in \xR \]

\SSsect Основные свойства \( \xe(t) \)~:
\[
\xe(t+t') = \xe(t) \cdot \xe(t')
\]
\[
\xe(kt) = \xe(t)^k
\]
\[
\xe(t+1) = \xe(t)
\]
\[
|\xe(t)| = 1
\]
\[
\xe(-t) = \xe(t)^{-1} = \overline{\xe(t)}
\]

\SSsect В силу периодичности, \( \xe(t) \) может быть определена для аргумента \( t \) из \( \xR/\xZ \). 
Фактически, \( \xe(t) : \xR/\xZ \simeq \xT \).

\end{document}
