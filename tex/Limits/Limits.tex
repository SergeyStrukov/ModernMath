% Limits.tex
% Copyright (c) Sergey Strukov. All rights reserved. This is a public document. You can freely distribute and use it, providing the authorship and the copyright note is unchanged.

\input ../SSCommon.tex

\begin{document}
\selectlanguage{russian}
    
\SScover
    
\SStitle{Пределы}

Исторически, теория пределов возникла как часть анализа. 
Однако, после появления топологии, она в обобщённой форме стала естественной частью топологии. 
В этой статье теория пределов излагается в законченной геометрической форме. 
Кратко говоря, предел --- это продолжение функции по непрерывности на специальных топологических пространствах --- \mbox{\textbf{фильтрах}}. 
Подобная конструкция делает большинство свойств пределов \textit{наглядно очевидными}.

\vspace
    
\SSbullet 

\SSsect[def] Пусть \( F \) --- топологическое пространство. 
\( F \) называется \underline{фильтром}, если все точки \( F \), кроме одной, открыты. 
Такая точка, тавтологически, определена однозначно. 
Будем обозначать её \( \infty_F \). 
Следуя общему правилу, положим \( F^\circ := F\backslash\{\infty_F\} \).

\SSsect Пусть \( F \) --- \underline{фильтр}. 

Тогда справедливы следующие утверждения:
\begin{itemize}[label=]
\item \( F^\circ \) открыто, но не замкнуто,
\item \( F^\circ \) дискретно,
\item \( \overline{F^\circ} = F ~,~ F^\circ\neq\varnothing \) ,
\item \( \infty_F \) замкнута,
\item \( F \) недискретно.
\end{itemize}

\pagebreak

\SSsect[def] Пусть \( F \) --- фильтр, \( X \) --- топологическое пространство, 
\mbox{\( f: F^\circ \rightarrow X \) --- любое отображение}. 

Тогда \( f \) непрерывно (потому что \( F^\circ \) дискретно!).

Пусть \( x \in X \), \( g \) --- продолжение \( f \) на \( F \), такое, что \( g(\infty_F) = x \) .

Если \( g \) непрерывно, то будем говорить, что \( \displaystyle x = \lim_{F} f\) .

\begin{tikzpicture}
\SSDiag[M]
{ 
    F^\circ & F & \infty_F & g(\infty_F) = x \\
            & X \\
};
\path
(M-1-1) edge [->] node [above] {\( \subset \)} (M-1-2)
(M-1-1) edge [->] node [left] {\( f \)} (M-2-2)
(M-1-2) edge [->,dashed] node [right] {\( g \)} (M-2-2)
(M-1-3) edge [->] (M-1-2) 
(M-1-3) edge [->] node [right] {\( x \)} (M-2-2) ;
\end{tikzpicture}

\SSsect Пусть \( F \) --- фильтр, \( p: X \rightarrow Y \) --- непрерывное отображение топологических пространств, \( f: F^\circ \rightarrow X \) --- отображение.

Тогда \( \displaystyle x = \lim_{F} f ~\Rightarrow~ \displaystyle p(x) = \lim_{F} p\circ f \) .

\SSsect Пусть \( F \) --- фильтр, \( X \) --- топологическое пространство, \( \left\lbrace Y_i \right\rbrace_{i \in I} \) --- семейство топологических пространств, \( \left\lbrace p_i : X \rightarrow Y_i \right\rbrace_{i \in I} \) --- семейство непрерывных отображений. 
Пусть топология \( X \) порождена семейством \( \left\lbrace p_i \right\rbrace_{i \in I} \) . 
Пусть \(  f: F^\circ \rightarrow X ~,~ x \in X \) .

Тогда, \( \displaystyle x = \lim_{F} f ~\Leftrightarrow~ \forall~i \in I~ p_i(x) = \lim_{F} p_i \circ f \) . 

\SSsect Пусть \( F \) --- фильтр, \( X \) --- топологическое пространство, 
\( f: F^\circ \rightarrow X \) --- отображение.

Пусть \( A \subset X \) замкнуто.

Если \( f\left( F^\circ \right) \subset A ~,~ \displaystyle x = \lim_{F} f \) , то \( x \in A \) .

\SSproof

Пусть \( g \) --- непрерывное продолжение \( f \), такое, что \( g(\infty_F) = x \) .

Тогда \( x \in g(F) = g(\overline{F^\circ}) \subset \overline{g(F^\circ)} = \overline{f(F^\circ)} \subset \overline{A} = A \) .

Тоже самое по-другому: \( F^\circ \subset g^{-1}(A) ~\Rightarrow~ \overline{F^\circ} \subset g^{-1}(A) ~\Rightarrow~ \infty_F \in g^{-1}(A) ~\Rightarrow~ x \in A \) . 

\SSendp

\SSsect[!!!] Пусть \( F \) --- фильтр, \( X \) --- \textbf{хаусдорфово} топологическое пространство, \( f: F^\circ \rightarrow X \) .

Тогда \( \exists \) не более одного предела \( \displaystyle x = \lim_{F} f \) .

Почему у этого пункта аж три восклицательных знака? 
Причина в том, что в математике очень часто пределы в хаусдорфовы пространства используются для \textbf{построения} объектов. 
Например: производная, интеграл, сумма ряда и.т.п. 
Для этого нужна единственность, обеспеченная этим свойством.

\SSproof

Пусть \( \displaystyle x,y = \lim_{F} f \) . 
Тогда \( \displaystyle (x,y) = \lim_{F}~(f,f) \) , но \( (f,f) : F^\circ \rightarrow X \times X \) отображает \( F^\circ \) в диагональ \( \Delta_X \subset X \times X \), которая, в силу хаусдорфовости, замкнута в \( X \times X \). 
Значит, \( (x,y) \in \Delta_X \) , т.е. \( x=y \) .

\SSendp

\pagebreak

\SSbullet 

\SSsect[def] Пусть \( F \) и \( G \) --- два фильтра. 
\( \tau : F \rightarrow G \) --- \underline{морфизм фильтров}, если \( \tau \) --- непрерывное отображение и \( \tau^{-1}(\infty_G)=\{\infty_F\} \).

Фильтры и их морфизмы образуют категорию.

\SSsect Пусть \( F \) и \( G \) --- два фильтра. Если \( \tau : F \rightarrow G \) (морфизм фильтров), то:
\begin{itemize}[label=]
\item \( \tau(F^\circ) \subset G^\circ \) ,
\item \( \tau(\infty_F)=\infty_G \) , 
\item \( \tau^\circ: F^\circ \rightarrow G^\circ \) --- отображение, индуцированное \( \tau \) .
\end{itemize}

\SSsect Пусть \( F \) и \( G \) --- два фильтра.
Пусть \( \tau^\circ: F^\circ \rightarrow G^\circ \) --- отображение. 
Тогда \(  \tau^\circ \) может быть продолжено до морфизма фильтров \( \tau : F \rightarrow G \) (очевидно, единственным образом) \( \Leftrightarrow~ \displaystyle \infty_G = \lim_{F} \tau^\circ \) .

\SSsect Пусть \( F \) и \( G \) --- два фильтра, \( X \) --- топологическое пространство, \( \tau : F \rightarrow G \) , \( f:G^\circ \rightarrow X \) .

Тогда если \( \displaystyle x = \lim_{G} f \) , то \( \displaystyle x = \lim_{F} f\circ \tau^\circ \) .

\pagebreak

\SSbullet 

\SSsect Пусть \( G \) --- фильтр, \( F \) --- топологическое пространство, \( \tau : F \rightarrow G \) --- непрерывное \textbf{инъективное} отображение.

Тогда \( F \) дискретно \( \Leftrightarrow~ \tau^{-1}(\infty_G) \) открыто.

Если \( F \) недискретно, то \( F \) --- фильтр и \( \tau \) --- морфизм фильтров. 

\SSproof

Пусть \( x \in F \). \( \tau \) инъективно, значит, \( \{x\} = \tau^{-1}(\{\tau(x)\}) \) .

Поэтому если \( \tau(x) \neq \infty_G \), то \( x \) открыта.
Если \( \tau(x) = \infty_G \), то \( \{x\} = \tau^{-1}(\infty_G) \) .

Таким образом, \( \tau^{-1}(\infty_G) \) открыто \( \Rightarrow \) все точки \( F \) открыты \( \Rightarrow \) \( F \) дискретно.

Если \( F \) дискретно, то \( \tau^{-1}(\infty_G) \) , тавтологически,  открыто.

Если \( F \) не дискретно, то \( \tau^{-1}(\infty_G) \) не открыто.
В частности, это множество непусто, значит, состоит из одной точки \( a \). Если \( x \neq a \) , то \( x \) открыта, \( a \) не открыта. Т.е. \( F \) --- фильтр и \( \tau \) --- морфизм фильтров.

\SSendp

\SSsect Пусть \( G \) --- фильтр, \( F \subset G \) --- подпространство. Тогда \( F \) дискретно \( \Leftrightarrow~ \{\infty_G\} \bigcap F \) открыто в \( F \).

Если \( F \) недискретно, то \( F \) --- фильтр и \( F \subset G \) --- морфизм фильтров.

Такие \( F \) называются подфильтрами.

\SSsect Пусть \( (F, \cT ) \) --- фильтр. 

Пусть \( \cS \) --- более тонкая топология на \( F \) , чем \( \cT \) .

Тогда \( \cS \) дискретна \( \Leftrightarrow~ \{\infty_F\} \in \cS \) .

Если \( \cS \) недискретна, то \( (F, \cS ) \) --- фильтр и \( (F, \cS ) \rightarrow (F, \cT ) \) --- морфизм фильтров.

Такие \( (F, \cS ) \) называются более тонкими фильтрами.

\SSsect[def] \( F \) --- \underline{ультрафильтр}, если любой более тонкий фильтр \( F' = F \) .

Другими словами, любая, более тонкая топология на \( F\) или дискретна, или совпадает с исходной.

\SSsect[!] Пусть \( F \) --- фильтр. Тогда \( \exists \) более тонкий \underline{ультра}фильтр.

\SSproof

Пусть \( F = (F,\cT) \) .

\( \Lambda := \{ \cS | \cS \) --- топология на \( F, \cS \supset \cT , \{\infty_F\} \notin \cS \} \) .

Если \( \cS \in \Lambda \) , то \( (F,\cS) \) есть более тонкий, чем \( F \) фильтр, и все такие фильтры получаются этим способом.

\( \Lambda \) частично упорядочено включением.

Если \( \cS \in \Lambda \) --- максимальный элемент, то \( (F,\cS) \) и есть требуемый более тонкий ультрафильтр.

Поэтому достаточно доказать, что \( \Lambda \) индуктивно и воспользоваться леммой Цорна.

Пусть \( \Sigma \subset \Lambda \) --- цепь. \( \cB := \bigcup \Sigma \) .

Тогда, \( F \in \cB \) , \( \{\infty_F\} \notin \cB \) , \( A,B \in \cB  \Rightarrow A \bigcap B \in \cB \) .

Значит, \( \cB \) --- мультипликативный базис топологии \( \cS \) .

Ясно, что \( \cS \in \Lambda \) и что \( \cS \) мажорирует \( \Sigma \) .

\SSendp

\pagebreak

\SSbullet 

\vspace

\begin{center}
Пусть \( F \) --- фильтр,

\( X \) --- топологическое пространство,

\( f: F^\circ \rightarrow X \) .
\end{center}

\SSsect Пусть \( \Gamma := \{~ (t,f(t)) ~|~ t \in F^\circ ~\} \subset F \times X \) --- график \( f \) .

Пусть \( x \in X \) , \( \Gamma_x := \Gamma \cup \{ (\infty_F,x) \} \subset F \times X \) --- график продолжения \( f \) , такого, что \( \infty_F \mapsto x \) .
\vspace
\begin{tikzpicture}
\draw[] (0,0) node[anchor=north] {\(F\)} -- (6,0) ;
\draw[] (3,0.1) -- (3,-0.1) node[anchor=north] {\(\infty_F\)} ;
\draw[] (7,1) -- (7,7) node[anchor=west] {\(X\)} ;
\draw[dashed] (3,1) -- (3,7) ;
\draw[] (6.9,6) -- (7.1,6) node[anchor=west] {\(x\)} ;
\draw[] (3,6) node[anchor=east] {\(\Gamma_x\)} ;
\draw[] (3,6) node {\(\bullet\)} ;
\draw[dashed] (3.2,6) -- (6.8,6) ;
\draw[] plot [smooth] coordinates { (0,1) (1,3) (2,2) (2.5,4) } ;
\draw[] plot [smooth] coordinates { (3.5,5.5) (4,2.5) (5,4) (5.5,1) } ;
\draw[] (2,4) node[] {\(\Gamma\)} ;
\end{tikzpicture}

\SSsect Рассмотрим следующую диаграмму:

\begin{tikzpicture}
\SSDiag[M]
{ 
   \Gamma_x & X \\
   F \\
};
\path
(M-1-1) edge [->] node [above] {\( pr_X \)} (M-1-2)
(M-1-1) edge [->] node [right] {\( pr_F \) биективно} (M-2-1) ;
\end{tikzpicture}

Тогда
\[ x = \lim_{F} f \Leftrightarrow pr_F \mbox{~изоморфизм} \Rightarrow \Gamma_x \mbox{~недискретно} \]

\SSsect[def] \( x \in X \) называется предельной точкой \( f \) относительно \( F \) , если \( \Gamma_x \) недискретно. Множество предельных точек \( f \) относительно \( F \) будем обозначать \( \underset{F}{Lim}(f) \) .

\SSsect \( x = \lim\limits_{F} f \Rightarrow x \in \underset{F}{Lim}(f) \) .

\pagebreak

\SSsect Cледующие утверждения эквивалентны:

\begin{enumerate}[label={\alph*)}]
\item \( \Gamma_x \) дискретно 
\item \( (\infty_F,x) \) открыто в \( \Gamma_x \) 
\item \( \Gamma \) замкнуто в \( \Gamma_x \) 
\item \( \overline{\Gamma} \bigcap \Gamma_x = \Gamma \)  
\item \( (\infty_F,x) \notin \overline{\Gamma} \)  
\item \( \exists \) окрестности \( \infty_F \in V, x \in U\) , такие, что \( V \times U \bigcap \Gamma = \varnothing \) 
\item \( \exists \) окрестности \( \infty_F \in V, x \in U\) , такие, что \( V \bigcap f^{-1}(U) = \varnothing \)  
\end{enumerate}

\SSsect \( \overline{\Gamma} \bigcap \infty_F \times X = \infty_F \times \underset{F}{Lim}(f) \) .
\( \underset{F}{Lim}(f) \) замкнуто в \( X \) .

\SSsect \( \underset{F}{Lim}(f) = \bigcap\limits_{\infty_F \in V} \overline{f(V^\circ)} \)

\SSsect Пусть \( \tau : G \rightarrow F \) --- морфизм фильтров. Тогда \( \underset{G}{Lim}(f \circ \tau^\circ) \subset \underset{F}{Lim}(f) \) .

\SSproof
\vspace
\begin{tikzpicture}
\draw[] node[anchor=north] {\(G\)} (0,0) -- (6,0) ;
\draw[] (3,0.1) -- (3,-0.1) node[anchor=north] {\(\infty_G\)} ;
\draw[] (7,1) -- (7,7) node[anchor=west] {\(X\)} ;
\draw[dashed] (3,1) -- (3,7) ;
\draw[thick] (3,4) -- (3,6) ;
\draw[] plot [smooth] coordinates { (0,1) (1,3) (2,2) (2.5,4) } ;
\draw[] plot [smooth] coordinates { (3.5,5.5) (4,2.5) (5,4) (5.5,1) } ;
\draw[] (2,4) node[] {\(\Gamma'\)} ;
\draw[] (8,4) node[] {\(\rightarrow\)} ;
\draw[] (9,0) node[anchor=north] {\(F\)} -- (15,0) ;
\draw[] (12,0.1) -- (12,-0.1) node[anchor=north] {\(\infty_F\)} ;
\draw[] (16,1) -- (16,7) node[anchor=west] {\(X\)} ;
\draw[dashed] (12,1) -- (12,7) ;
\draw[thick] (12,4) -- (12,6) ;
\draw[] plot [smooth] coordinates { (9,1) (10,3) (11,2) (11.5,4) } ;
\draw[] plot [smooth] coordinates { (12.5,5.5) (13,2.5) (14,4) (14.5,1) } ;
\draw[] (11,4) node[] {\(\Gamma\)} ;
\end{tikzpicture}

\[ (\tau \times id_X)(\Gamma') \subset \Gamma \]
\[ (\tau \times id_X)(\overline{\Gamma'}) \subset \overline{\Gamma} \]

\SSendp

\pagebreak

\SSsect[!] Следующие утверждения равносильны:
\begin{enumerate}[label={\alph*)}]
\item \( x \in \underset{F}{Lim}(f) \)
\item \( \Gamma_x \) есть фильтр
\item \( \exists F' \) --- более тонкий фильтр, чем \( F \) , такой, что \( x = \lim\limits_{F'} f \)
\item \( \exists F' \) --- более тонкий фильтр, чем \( F \) , такой, что \( x \in \underset{F'}{Lim}(f) \)
\end{enumerate}

\SSsect Пусть \( F \) --- \textbf{ультрафильтр}. Тогда \( \underset{F}{Lim}(f) = \lim\limits_{F} f \) .

\SSsect Пусть \( X \) хаусдорфово. Тогда \( x = \lim\limits_{F} f ~\Rightarrow~ \underset{F}{Lim}(f) = \{x\} \) .

\SSsect Пусть \( p : X \rightarrow Y \) --- непрерывное отображение. Тогда \( p(\underset{F}{Lim}(f)) \subset \underset{F}{Lim}(f \circ p) \) .

\pagebreak

\SSbullet 

\SSsect[def] Пусть \( F \) --- фильтр. \textit{Базой} \( F \) называется базис окрестностей в \( \infty_F \) . Т.е. это семейство \( \cB \), такое, что \( \forall~U \in \cB \) есть окрестность \( \infty_F \) и \( \forall~V \) , окрестности \( \infty_F \), \( \exists~V \in \cB \) , такая, что \( V \subset  U \) .

\SSsect Пусть \( F \) --- фильтр, \( X \) --- топологическое пространство, \( f: F \rightarrow X \) --- функция. Тогда \( f \) непрерывна \( \Leftrightarrow f \) непрерывна в точке \( \infty_F \) .

\SSsect Пусть \( F \) --- фильтр, \( \cB \) --- база \( F \) , \( X \) --- топологическое пространство, \( f: F \rightarrow X \) --- функция, \( f(\infty_F)=x \) . Тогда \( f \) непрерывна \( \Leftrightarrow \forall~x \in U \) окрестности \( \exists~V \in \cB \) , такая, что \( f(V) \subset U \).

\SSsect[!] Пусть \( X \) --- топологическое пространство, \( A \subset X \) , \( x \in \overline{A} \) . Тогда \( \exists \) фильтр \( F \) и функция \( f: F^\circ \rightarrow X \) , такие, что \( x = \lim\limits_{F} f \) .

\SSproof

Положим \( F = A \bigsqcup \{\infty\} \) . Топологию на \( F \) зададим следующим базисом:
\begin{itemize}[label=]
\item \( \{a\} \) в точке \( a \in A \) ,
\item \( \{~( U \bigcap A ) \cup \{\infty\} ~|~ x \in U \) --- окрестность \( ~\} \) , в точке \( \infty \) .
\end{itemize}

Тогда \( F \) --- фильтр, \( \infty_F=\infty \) , \( F^\circ=A \) . Определим \( f:=id_A \) , \( g: F \rightarrow X \) --- продолжение \( f \) , такое, что \( g(\infty)=x \) . Тогда для любой окрестности \( x \in U \) будем иметь \( g(( U \bigcap A ) \cup \{\infty\}) \subset U \) , значит, \( g \) непрерывна, поэтому \( x = \lim\limits_{F} f \) .

\SSendp

\SSsect[def] \( (\Lambda,\rightarrow) \) --- направленное множество, если:
\begin{itemize}[label=]
\item \( \Lambda \) --- непустое множество,   
\item \( \rightarrow \) --- отношение на \( \Lambda \) , удовлетворяющее следующим условиям:
\begin{enumerate}[label={\alph*)}]
\item \( \lambda \rightarrow \lambda \) (рефлексивность),
\item \( \lambda \rightarrow \mu ~\&~ \mu \rightarrow \nu \Rightarrow \lambda \rightarrow \nu \) (транзитивность),
\item \( \forall~\lambda,\mu ~\exists~ \nu \) , такое, что \( \lambda \rightarrow \nu ~\&~ \mu \rightarrow \nu \)
\end{enumerate}
\end{itemize}

\( \Lambda \) можно рассматривать как малую категорию, где \( \Lambda \) есть множество объектов,

\[ Hom(\lambda,\mu)=
   \begin{cases}
   \{\varnothing\}, & \text{если} ~\lambda \rightarrow \mu \\
   \varnothing    , & \text{в противном случае}
   \end{cases}
\]

\pagebreak

\SSsect[def] Пусть \( \Lambda \) --- направленное множество.
\[ \Lambda \rightarrow \infty ~:=~ \Lambda \sqcup \{\infty\} \]
Введём на \( \Lambda \rightarrow \infty \) топологию со следующим базисом:
\begin{itemize}[label=]
\item \( \{\lambda\} \) в точке \( \lambda \in \Lambda \) ,
\item \( \{~\{~\mu~|~\lambda\rightarrow\mu~\} \cup \{\infty\} ~|~ \lambda \in \Lambda ~\} \) , в точке \( \infty \) .
\end{itemize}
Тогда \( \Lambda \rightarrow \infty \) есть фильтр, \( (\Lambda \rightarrow \infty)^\circ=\Lambda \) . Самый распространённый пример --- \( \xN \rightarrow \infty \) .

\SSsect Пусть \( \Lambda \) и \( \Lambda' \) --- направленные множества, \( \tau: \Lambda \rightarrow \Lambda' \) --- монотонное отображение. Т.е. \( \lambda \rightarrow \mu \Rightarrow \tau(\lambda) \rightarrow \tau(\mu) \) . \( \tau \) можно рассматривать как функтор, если интерпретировать \( \Lambda \) и \( \Lambda' \) как малые категории.
Тогда \( \tau \) продолжается до морфизма фильтров \( (\Lambda \rightarrow \infty) \rightarrow (\Lambda' \rightarrow \infty) \) (всегда однозначно) \( \Leftrightarrow  \forall~\lambda' \in \Lambda' ~\exists~\lambda \in \Lambda ~,~ \lambda' \rightarrow \tau(\lambda) \) .

\SSproof

\( \tau \) продолжается до морфизма фильтров \( \Leftrightarrow \lim\limits_{\Lambda \rightarrow \infty} \tau = \infty \) . Это значит, что \( \forall~\lambda' \in \Lambda' ~\exists~\lambda \in \Lambda ~,~ \lambda \rightarrow \mu \Rightarrow \lambda' \rightarrow \tau(\mu) \) . В частности, \( \lambda' \rightarrow \tau(\lambda) \) . Наоборот, пусть \( \lambda' \rightarrow \tau(\lambda) \) . Тогда \( \lambda \rightarrow \mu \Rightarrow \tau(\lambda) \rightarrow \tau(\mu) \Rightarrow \lambda' \rightarrow \tau(\mu) \) .

\SSendp

\SSsect Пусть \( \Lambda \) и \( \Lambda' \) --- направленные множества. Тогда \( \Sigma := \Lambda \times \Lambda' \) со следующим отношением порядка: \( (\lambda,\lambda') \rightarrow (\mu,\mu') \Leftrightarrow \lambda \rightarrow \mu ~\&~  \lambda' \rightarrow \mu' \) , есть направленное множество. При этом проекции \( \Sigma \rightarrow \Lambda \) и \( \Sigma \rightarrow \Lambda' \) монотонны и продолжаются до морфизмов фильтров \( (\Sigma \rightarrow \infty) \rightarrow (\Lambda \rightarrow \infty) \) и \( (\Sigma \rightarrow \infty) \rightarrow (\Lambda' \rightarrow \infty) \) .

\SSsect Пусть \( \Lambda \) --- направленное множество, \( \lambda_{max} \) --- наибольший элемент \( \Lambda \) , т.е. \( \forall~\lambda \in \Lambda~ \lambda \rightarrow \lambda_{max} \) . Это равносильно тому, что \( \lambda_{max} \) --- максимальный элемент \( \Lambda \) , т.е. \( \forall~\lambda \in \Lambda~ \lambda_{max} \rightarrow \lambda \Rightarrow \lambda \rightarrow \lambda_{max} \) .
Тогда \( \forall~ f:\Lambda \rightarrow X \) , где \( X \) --- топологическое пространство,
\[ f(\lambda_{max}) = \lim\limits_{\Lambda \rightarrow \infty} f \]

\SSsect Пусть \( \Lambda \) --- направленное множество, \( F \) --- фильтр, \( \tau:\Lambda \rightarrow F^{\circ} \) --- отображение. Тогда \( \tau \) продолжается до морфизма фильтров \( (\Lambda \rightarrow \infty) \rightarrow F \) (всегда однозначно) \( \Leftrightarrow \forall~ \infty_F \in U \) окрестности \( \exists~\lambda \in \Lambda ~\forall~\lambda \rightarrow \mu ~,~ \tau(\mu) \in U \) .

\pagebreak

\SSbullet 
\begin{center}
    \( X \) --- топологическое пространство
\end{center}

\SSsect[!] Следующие утверждения равносильны:
\begin{enumerate}[label={\alph*)}]
\item \( X \) компактно,
\item \( \forall~ Y ~ pr_Y: Y \times X \rightarrow Y \) замкнуто,
\item \( \forall \) фильтра \( F \) и \( f:F^{\circ} \rightarrow X ~,~ \underset{F}{Lim}(f) \neq \varnothing \) ,
\item \( \forall \) фильтра \( F \) и \( f:F^{\circ} \rightarrow X ~\exists \) более тонкий фильтр \( F' \) , такой, что существует \( \lim\limits_{F'} f \) ,
\item \( \forall \) \textbf{ультрафильтра} \( F \) и \( f:F^{\circ} \rightarrow X ~\exists~ \lim\limits_{F} f \)
\end{enumerate}

\SSproof

\underline{a) \( \Rightarrow \) b)}
\vspace

Пусть \( A \subset Y \times X \) --- замкнуто, тогда \( U = Y \times X \setminus A \) открыто.

\begin{tikzpicture}
\draw[] (0,0) node[anchor=east] {\(Y\)} -- (6,0) ;
\draw[] (3,0.1) -- (3,-0.1) node[anchor=north] {\(y\)} ;
\draw[thick] (4,0) -- (5.5,0) ;
\draw[] (4.75,0) node[anchor=north] {\(B\)} ;
\draw[] (7,1) -- (7,6) node[anchor=south] {\(X\)} ;
\draw[] (0,1) rectangle (6,6) ;
\draw[dashed] (3,1) -- (3,6) ;
\draw[] (1.5,4) node[] {\(U\)} ;
\draw[] (4.75,4) node[] {\(A\)} ;
\draw[] (4.75,4) circle (0.75) ;
\draw[] (2.5,1) rectangle (3.5,2) ;
\draw[] (2.3,1.9) rectangle (3.7,3) ;
\draw[] (2.7,2.8) rectangle (3.3,4.5) ;
\draw[] (2.6,4.4) rectangle (3.4,5) ;
\draw[] (2,4.6) rectangle (4,6) ;
\end{tikzpicture}

Положим \( B = pr_Y(A) \) , \( C = Y \setminus B \) . Нам надо доказать, что \( B \) замкнуто, это равносильно тому, что \( C \) открыто.
\[ C = \{~y \in Y~|~y \times X \bigcap A = \varnothing ~\} = \{~y \in Y~|~y \times X \subset U ~\} \]
Пусть \( y \in C \) , тогда \( y \times X \subset \underset{i \in I}{\bigcup} W_i \times V_i \) , где \( W_i \) --- окрестность \( y \), \( V_i \) --- открыто в \( X \) и \( W_i \times V_i \subset U \) . Но \( y \times X \) компактно, следовательно можно считать \( I \) конечным (заменяя на конечное подпокрытие). Тогда \( W = \underset{i \in I}{\bigcap} W_i \) --- окрестность \( y \) и \( X \subset \underset{i \in I}{\bigcup} V_i \) . Значит, \( W \times X = \underset{i \in I}{\bigcup} W \times V_i \subset \underset{i \in I}{\bigcup} W_i \times V_i \subset U \) . Поэтому \( W \subset C \) . Итак, \( C \) открыто.

\vspace

\underline{b) \( \Rightarrow \) c)}
\vspace

Пусть \( \Gamma \) --- график \( f \) .

\begin{tikzpicture}
    \draw[] (0,0) node[anchor=north] {\(F\)} -- (6,0) ;
    \draw[] (3,0.1) -- (3,-0.1) node[anchor=north] {\(\infty_F\)} ;
    \draw[] (7,1) -- (7,7) node[anchor=west] {\(X\)} ;
    \draw[dashed] (3,1) -- (3,7) ;
    \draw[] plot [smooth] coordinates { (0,1) (1,3) (2,2) (2.5,4) } ;
    \draw[] plot [smooth] coordinates { (3.5,5.5) (4,2.5) (5,4) (5.5,1) } ;
    \draw[] (2,4) node[] {\(\Gamma\)} ;
\end{tikzpicture}

Имеем, \( pr_F(\overline{\Gamma}) \) замкнуто, содержит \( F^{\circ} \) , значит, \( \supset \overline{F^{\circ}}=F \) . Тем самым, \( \infty_F \in pr_F(\overline{\Gamma}) \Rightarrow \infty_F\times X \bigcap \overline{\Gamma} \neq \varnothing \Rightarrow \underset{F}{Lim}(f) \neq \varnothing \) .

\vspace
\underline{c) \( \Rightarrow \) a)}
\vspace

Пусть \( \{U_i\}_{i \in I} \) --- открытое покрытие \( X \) , из которого нельзя выбрать конечное подпокрытие. \( \Lambda:=\{~J \subset I~|~J~ \text{конечно} ~\} \) . Упорядочим \( \Lambda \) по включению: \( J \rightarrow J' \) , если \( J \subset J' \) . Тогда \( \Lambda \) --- направленное множество. Для любого \( J \in \Lambda \) выберем \( f(J) \in X \setminus \underset{i \in J}{U_i} \) .

Пусть \( i \in I \) , \( J=\{i\} \) . Пусть \( J \rightarrow J' \) , тогда \( i \in J' \) , значит, \( f(J') \notin U_i \) . Т.е. \( f(\{~J'~|~J \rightarrow J'~\}) \bigcap U_i = \varnothing \Rightarrow \underset{\Lambda \rightarrow \infty}{Lim}(f) \bigcap U_i = \varnothing \) . Поскольку это верно для любого \( i \) , \( \underset{\Lambda \rightarrow \infty}{Lim}(f) = \varnothing \) , противоречие.

\vspace
Импликации \underline{c) \( \Rightarrow \) d) \( \Rightarrow \) e) \( \Rightarrow \) c)} тривиальны.

\SSendp

\SSsect[!!!] (Теорема Тихонова о произведениях).
Если \( \{X_i\}_{i \in I} \) --- семейство компактных пространств, то их произведение \( \underset{i \in I}{\prod} X_i \) тоже компактно.

\pagebreak

\SSbullet 

\SSsect Пусть \( F \) --- фильтр. Тогда \( F \) есть ультрафильтр тогда, и только тогда, когда для любого разбиения \( F^{\circ} = A \bigsqcup B \) верно, что \( A \bigcup \{\infty_F\} \) или \( B \bigcup \{\infty_F\} \) открыты (но не оба вместе).

\SSproof
\vspace
TODO
\vspace
\SSendp

\SSsect Пусть \( X \) --- топологическое пространство, \( A \subset X \) , \( x \in \overline{A} \) . Пусть \( X \) обладает счётным базисом окрестностей в точке \( x \) . Тогда \( \exists~f:\xN \rightarrow A \) , такая, что \( x = \lim\limits_{\xN \rightarrow \infty} f \) .

\SSproof

Пусть \( \{U_n\}_{n \in \xN} \) --- убывающий базис окрестностей в точке \( x \) . Тогда \( \forall~n \in \xN~U_n\bigcap A \neq \varnothing \) . Выберем для всех \( n \in \xN \) точку \( f(n) \in U_n\bigcap A \) . Тогда \( f:\xN \rightarrow A \) . Далее, \( f(k) \in U_n \) , если \( k \geqslant n \) . Т.е. \( f(\{~k \in \xN~|~k \geqslant n~\}) \subset U_n \) . Значит, \( x = \lim\limits_{\xN \rightarrow \infty} f \) .

\SSendp

\SSsect Пусть \( F \) --- фильтр со счётной базой фильтра. Тогда существует морфизм фильтров \( (\xN \rightarrow \infty) \rightarrow F \) .

\SSproof

\( \infty_F \in \overline{F^{\circ}} \)

\SSendp

\SSsect Пусть \( F \) --- фильтр со счётной базой фильтра. Пусть \( A \subset F \) не замкнуто. Тогда существует морфизм фильтров \( \tau: (\xN \rightarrow \infty) \rightarrow F \) , такой, что \( \tau^{-1}(A) = \xN \) .

\SSproof

Достаточно заметить, что \( \infty_F \notin A \) и \( A \bigcup \{\infty_F\} \) --- подфильтр \( F \).

\SSendp

\SSsect Пусть \( F \) --- фильтр со счётной базой фильтра. Пусть \( X \) --- топологическое пространство, \( f:F^{\circ} \rightarrow X \) , \( x \in X \) . Пусть для любого морфизма фильтров \( \tau: (\xN \rightarrow \infty) \rightarrow F \) верно, что \( x =  \lim\limits_{\xN \rightarrow \infty} f \circ \tau \) . Тогда \( x =  \lim\limits_{F} f \) .

\SSproof

TODO

\SSendp

\end{document}
