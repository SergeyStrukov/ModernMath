% Limits.tex
% Copyright (c) Sergey Strukov. All rights reserved. This is a public document. You can freely distribute and use it, providing the authorship and the copyright note is unchanged.

\input ../SSCommon.tex

\begin{document}
\selectlanguage{russian}
    
\SScover
    
\SStitle{Пределы}

Исторически, теория пределов возникла как часть анализа. Однако, после появления топологии она в обобщённой форме стала естественной частью её. 
В этой статье теория пределов излагается в законченной геометрической форме. 
Кратко говоря, предел --- это продолжение функции по непрерывности на специальных топологических пространствах --- \underline{фильтрах}. 
Подобная конструкция делает большинство свойств пределов наглядно очевидными.
    
\SSbullet 

\SSsect Пусть \( F \) --- топологическое пространство. \( F \) называется \underline{фильтром}, если все точки \( F \), кроме одной, открыты. 
Такая точка, тавтологически, определена однозначно. 
Будем обозначать её \( \infty_F \). 
Следуя общему правилу, положим \( F^\circ := F\backslash\{\infty_F\} \).

\SSsect Пусть \( F \) --- \underline{фильтр}. Тогда справедливы следующие утверждения:
\begin{itemize}[label=]
\item \( F^\circ \) открыто, но не замкнуто
\item \( F^\circ \) дискретно
\item \( \overline{F^\circ} = F ~,~ F^\circ\neq\varnothing \)
\item \( \infty_F \) замкнута
\item \( F \) недискретно
\end{itemize}

\end{document}
